%%%%%%%%%%%%%%%%%%%%%%%%%%%%%%%%%%%%%%%%%%%%%%%%%%%%%%%%%
% app1_terms.tex
% Appendix 1: Derivation of SAR notation.
% This section: Charles Twardy
%%%%%%%%%%%%%%%%%%%%%%%%%%%%%%%%%%%%%%%%%%%%%%%%%%%%%%%%%

%\section{Derivation of SAR notation}
\label{sec:terms}

If your background is Search \& Rescue, you may be wondering how these
formulas relate to familiar concepts like POA, POS, and POD.  In the
first part of this appendix, we will derive the function to optimize
from the basic concepts taught in SAR courses. The setup is easy, but
the final steps are messy to write.  In the second part, we will
derive the from basic concepts in conditional probability. That
derivation is nicer, but not nearly as clean as the very abstract
treatment given in Section \ref{sec:form}, which is why these are in
the Appendix. First we define our variables.

\subsection{Background}
\label{sec:Background}

The basic equation governing search probability theory\footnote{There
  is no new material in this section. We are merely laying out search
  theory, as can be found in standard SAR references like
  \cite{frost99:_principles, syrotuck75:_introduction,
    cooper00_adjpoa, cooper00_defs}. However, we
do so in a manner convenient for subsequent derivations.} states that the
probability of success $(POS)$ of any search assignment is equal to
the product of two other probabilities: the probability that the
subject is in the area $(POA)$ and the probability of detecting the
subject if they are in that area $(POD)$.
\begin{equation}
  \label{eq:POS}
  POS = POA \times POD
\end{equation}
For now we shall just accept this. It can be derived from a few simple
considerations about search and Bayes' theorem, but we leave that to
the section on conditional probabilities.

Once an area is searched (without
success), the subject is less likely to be there, in proportion to how 
well the area was searched. In other words, after a search, the $POA$ of 
the region searched decreases. It can also be shown that the usual
formula is correct:
\begin{equation}
  \label{eq:POA_new}
  POA_{new} = POA_{old} - POS
\end{equation}

It is easy to understand $POA$. Each region has a known area (in
square meters, for example), and a probability of area $POA$, the
probability that the subject is in that area.\footnote{Sometimes
  called the probability of containment, $POC$ since it is the
  probability that the area in question contains the subject.} The
$POA$ is usually determined by a Mattson consensus\footnote{Probably
  from Mattson's 1979 presentation to a NATO conference,
  \cite{mattson80:_estab}.}  (a fancy name for ``average'') from case
histories, known facts, and probable scenarios.  Here we will just
assume that it is given and correct. Often we combine these two items
to define the \emph{probability density}:
\begin{equation}
  \label{eq:2}
  Pden = \frac{POA}{area}
\end{equation}
Sometimes $Pden$ is written as greek letter $\rho$, since it looks
like a funny `p'.

It is a little more involved to get the $POD$ of a resource in an
area. Although search managers usually use various crude estimation
techniques (``If there were 100 milk jugs in the area, how many do you 
think you would have found?''), search theory provides a better way to 
measure $POD$ based on how much effort was applied, among other things..

First, every resource can move with a certain speed $v$ in a certain
region and at that speed (in those conditions, looking for that kind
of object) have a certain
\emph{effective sweep-width}, usually denoted capital $W$.  $W$ is a
measure of how far that resource can expect to see and detect objects.
There are tables of sweep widths for many resources in many
conditions. We shall presume we have them available.\footnote{Alas,
  there are few or none for ground searchers yet. Until as a community 
  we can conduct some experiments, Frost has suggested in-field
  estimation using ``critical separation,'' being the distance
  a  searcher estimates he can see effectively enough to almost always
  spot targets if he were looking that way.}

Together $W$ and $v$ measure how quickly the resource can sweep
areas.\footnote{Indeed, sometimes people define \emph{effective
    sweep-rate}, $ESR = W v$. But we don't need an extra variable.}
We let $t$ be our measure of how much effort (resource-hours) has been
put in the region.

To get $POD$ we need to know how well the region has been
\emph{covered}. \emph{Coverage} is a measure of how thoroughly a given
amount of effort was able to search an area of a certain size. Higher
coverage means higher $POD$. For a given amount of a particular
resource's effort, a bigger area means less coverage. Likewise, for a
given amount of effort in a given area, a resource with a bigger $ESR$
will have more coverage. And finally, the longer we leave the resource
in a region (or the more of it we put there), the more area it sweeps,
and the higher the coverage. So, if we let $x$ be a measure of how
much of the resource's effort we put in an area, we may define a
resource's \emph{coverage}:
\begin{equation}
  \label{eq:C}
  C = \frac{Wvt}{\mathrm{area}}
\end{equation}

And $POD$ must be a function of coverage. It must be such that $POD$
starts at zero when we have zero coverage, and $POD$ can 
never exceed 1 no matter how much we cover the area. One function
which can do this is a negative exponential function. But we're not
just picking functions out of the air. It turns out that this is the
proper function for what is called \emph{random search}, and that
almost any real ground search is best modeled by the random-search
curve. \cite{stone89:_optimal} Therefore,
\begin{equation}
  \label{eq:3}
  POD = 1 - e^{-C}
\end{equation}
which can also be written
\begin{equation}
  \label{eq:9}
  POD = 1 - \exp(-C)
\end{equation}
(Since later on we will have many things in the exponent, it will be
easier to read if we use the `exp' form.)

\subsection{SAR notation}
\label{sec:SARnotation}

We have $R$ resources available for searching. Together, they give us
the total amount of available effort, $T$. The effort due to any
resource $r$ is $t_r$. It follows that 
\begin{equation}
  \label{eq:10}
  T = \sum_{r=1}^N t_r
\end{equation}

We also have $A$ areas to search. Each area $a$ has a certain
probability of area $POA_a$. Usually all probabilities initially sum
to 1, but since they do not really have to be normalized, the only
constraint is that: $(0 < \sum_{a=1}^A POA_a \leq 1)$.

Our problem is to figure out the best way to allocate all available
effort among the many areas. The effort  $t_r$ from any resource
may be split among any number of areas, but $r$ will always be used
completely, so $\sum_{a=1}^A t_{r,a} = t_r$. 

Conversely, several resources may combine their efforts in any one
area. But while there may be many areas left unsearched, we will never
fail to use a resource fully.

Coverage is defined in terms of resources and areas. If $W_{r,a}$ is
the sweep width for resource $r$ in area $a$, and the other variables
likewise, then the coverage in area $a$ by resource $r$ is:

\begin{equation}
  \label{eq:C_ra}
  C_{r,a} = \frac{W_{r,a} v_{r,a}}{A_a} t_{r,a}
\end{equation}

However, because the area sizes and the $W_{r,a}$ and $v_{r,a}$ remain fixed,
rather than continuing to write that entire fraction, we will
just define a new variable lowercase $w_{r,a} = \frac{W_{r,a} v_{r,a}}{A_a}$ and write:
\begin{equation}
  \label{eq:C_ra2}
  C_{r,a} = w_{r,a} t_{r,a}
\end{equation}

Once we know the coverage, we can find the $POD$ from equation \ref{eq:9}. 
\begin{equation}
  \label{eq:POD_ra}
  POD_{r,a} = 1 - \exp(-C_{r,a}) = 1 - \exp(-w_{r,a} t_{r,a})
\end{equation}

Now in the background section, we wrote the governing equation of
search theory $POS = POA \times POD$, and the associated equation for
updating the $POA$ of a region after an unsuccessful search:
$POA_{new} = POA_{old} - POS$. 

Since the second equation is just subtraction, we can apply the
formula repeatedly, and define the $POA$ for any sortie in terms of
the original $POA$ and the sum of all the $POS$ from all previous
searches. If we number our searches with an index, then the POA for
a search with resource $r$ is:
\begin{equation}
  \label{eq:POA2'}
  POA_{r} = POA_0 - \sum_{i=1}^{r-1} POS_i
\end{equation}


This generalizes quite easily to our notation. Imagine If the initial $POA$
for area $a$ is just $POA_a$, then to compute the current $POA$ facing
our next resource $r$ in area $a$, we sum over the
previously-allocated resources (up to resource $r-1$)\footnote{
  The order in which we consider resources for assignment is
  immaterial: all we want in the end is the right global allocation.
  So we might as well pick the resources straight off the list in
  order $1 \ldots r$. Which is how we have done it here.}:
\begin{equation}
  \label{eq:POA_ra}
  POA_{r,a} = POA_a - \sum_{i=1}^{r-1}POS_{i,a}
\end{equation}

Using the basic identity from Equation \ref{eq:POS} ($POS =
POA \times POD$), we can
rewrite Equation \ref{eq:POA_ra} in terms of products. Skipping the algebra:
\begin{eqnarray}
  \label{eq:15}
  \nonumber POA_{r,a} &=& POA_a \left[(1-POD_{1,a})\cdots(1-POD_{r-1,a})\right]\\
                   & & \\
  \nonumber &=& POA_a \left[\exp(w_{1,a}t_{1,a}) \cdots \exp(w_{r-1,a}t_{r-1,a})\right]\\
  \nonumber &=& POA_a \left[\exp\left(\sum_{i=1}^{r-1}w_{i,a}t_{i,a}\right)\right]
\end{eqnarray}
The second step is just substitution, and the final step is is just
replacing a product of exponents with an exponent of sums. At this point we are almost done.

Putting the $POS_{r,a}$ and the $POA_{r,a}$ together:
\begin{equation}
  \label{eq:POS2_ra}
  POS_{r,a} = POA_{r,a} \cdot POD_{r,a}
\end{equation}
which again comes from equation \ref{eq:POS}. 

And, in the end, we want to \emph{maximize} the \emph{total}
$POS$. The total $POS$ is just the sum over both $r$ and $a$. Writing
this out and substituting for $POA_{r,a}$ we find:
\begin{eqnarray}
  \label{eq:POS_tot}
  POS &=& \sum_{a=1}^A \sum_{r=1}^R POS_{r,a}\\
  \nonumber &=& \sum_{a=1}^A POS_{1,a} + POS_{2,a} + \cdots + POS_{R,a}\\
  \nonumber &=& \sum_{a=1}^A \left[
    POA_a POD_{1,a} +
    POA_a \exp(-w_{1,a}t_{1,a}) POD_{2,a} + \cdots + 
    POA_a \exp\left(-\sum_{i=1}^{R-1}w_{r,a}t_{r,a}\right) POD_{R,a} 
    \right]\\
   &=& \sum_{a=1}^A POA_a \left[
     POD_{1,a} +
     \exp(-w_{1,a}t_{1,a}) POD_{2,a}+ \cdots + 
     \exp\left(-\sum_{i=1}^{R-1}w_{r,a}t_{r,a}\right) POD_{R,a}
     \right]
\end{eqnarray}

Finally, if we expand the $POD_{r,a}$ terms we find that many of the
exponents cancel, leaving us with a much simpler form.

\fbox{
  \parbox{6in}{
  \begin{eqnarray}
    \label{eq:pos-final}
    \nonumber POS &=& \sum_{a=1}^A POA_a \left[
      (1-\exp(-w_{1,a}t_{1,a})) +
      \exp(-w_{1,a}t_{1,a}) (1-\exp(-w_{2,a}t_{2,a}))+ \cdots +\right. \\
    & & \left. \exp\left(-\sum_{r=1}^{R-1}w_{r,a}t_{r,a}\right)
      \left(1-\exp(-w_{R,a}t_{r,a})\right)
    \right]\\
    &=& \sum_{a=1}^APOA_a \left[
      1 - \exp\left(-\sum_{r=1}^R w_{r,a}t_{r,a}\right)
    \right]    
  \end{eqnarray}
  }
}

%%%%%%%%%%%%%%%%%%%%%%%%%%%%%%%%%%%%%%%%%%%%%%%%%%%%
\subsection{Conditional Probability Notation}
\label{sec:Cond-Prob-Notat}

The initial probability (before we get new evidence) for anything is
called its \emph{prior probability}. For example, we know from the
start that we will use all of our resources. Therefore the prior
probability for any resource is 1:
\begin{equation}
  \label{eq:p(r)}
  \Pr(R) = 1
\end{equation}

Each area has an initial (prior) probability of containing the
subject, which is usually called the $POA$ in search terms. We will
call is $\Pr(A)$.
\begin{equation}
  \label{eq:p(a)}
  \Pr(A) = POA
\end{equation}

The probability of something given that we know something else is
called the \emph{conditional probability}. For example, we might wish
to know the conditional probability of an area $A$, given that we have
chosen a particular resource $R$. This one is easy: the probability of
an area does not depend upon which resource we use
to search it, so it equals its initial probability:
\begin{equation}
  \label{eq:p(a|r)}
  \Pr(A|R) = P(A)
\end{equation}

For a resource searching in the area containing the subject, the
probability of detecting the subject is given by search
theory\footnote{This is the correct formula for the ``random'' or
  ``exponential'' detection profile, which is shown in \cite{} to be
  the most appropriate model for ground search.}
as $1 - e^{-C}$. Keeping with our notation in the previous section, we
use $t_{r,a}$ to denote the amount of effort from resource $r$ used in
area $a$, and $w_{r,a}$ to denote the effectiveness of the resource in
that area. So:
\begin{equation}
  \label{eq:4}
  \Pr(D|R,A) = 1 - e^{-w_{r,a} t_{r,a}} = 1 - \exp(-w_{r,a} t_{r,a})
\end{equation}

So what is the \emph{posterior} probability of an area? It is the
probability that the subject is in the area given that we have
searched but \emph{not detected} ($\no{D})$ the subject. If we have
$N$ resources, then by Bayes'
theorem (without renormalization):
\begin{eqnarray}
  \label{eq:a|-d}
  \Pr(A|\no{D}) &\propto& \Pr(A) \cdot \Pr(\no{D}|A)\\
  \nonumber     &=& \Pr(A) \cdot \left[\Pr(\no{D}|R_1,A) \ldots 
                   P(\no{D}|R_R,A)\right] \\
  \nonumber     &=& \Pr(A) \left[\Pr(\no{D}|R_1,A) \ldots
                   P(\no{D}|R_R,A)\right] \\
  \nonumber     &=& \Pr(A) \left[\exp(-w_{1,a} t_{1,a}) \ldots
                   e^{-w_{R,a} t_{R,a}} \right]\\
  \nonumber     &=& \Pr(A) \exp\left(-\sum_{r=1}^R w_{r,a} t_{r,a}\right)
\end{eqnarray} 

That gives us the updated probability for one area. And to get the
total probability that we have \emph{not} detected the subject up
until now, we just sum over the areas. By definition, this is just the
sum over areas of the \emph{joint probability} of being in each area
and being not-detected:
\begin{equation}
  \label{eq:6}
  \Pr(\no{D}) = \sum_A \Pr(\no{D},A)
\end{equation}
But the joint probability of $\no{D}$ and $A$ is by definition: 
\begin{equation}
  \label{eq:8}
  \Pr(\no{D},A) =  \Pr(A) \Pr(\no{D}|A)
\end{equation}
which is just the right-hand-side of Equation \ref{eq:a|-d}. We
infer that:
\begin{equation}
  \label{eq:-d|a}
  \Pr(\no{D}|A) = \exp\left(-\sum_{r=1}^R w_{r,a} t_{r,a}\right)
\end{equation}

Of course, in the end we want to maximize our chances of \emph{actually
  detecting} the subject, for the \emph{whole set of
  assignments}. Usually called the ``probability of success,'' this is
really just the overall $\Pr(D)$.

\fbox{
  \parbox{5in}{
    \begin{eqnarray}
      \label{eq:p(d)}
      \nonumber \Pr(D) &=& \sum_A \Pr(A) \Pr(D|A)\\
      \nonumber  &=& \sum_A \Pr(A) \left[ 1 - \Pr(\no{D}|A) \right]\\
      & & \\
      \nonumber  &=& \sum_A \Pr(A) \left[ 1 - \exp\left(-\sum_{r=1}^R w_{r,a}
          t_{r,a}\right) \right]
    \end{eqnarray}
    }
  }

Note that this agrees with Equation \ref{eq:pos-final}, the overall
$POS$ derived using conventional notation.


\end{document}

%%% Local Variables: 
%%% mode: latex
%%% TeX-master: "master"
%%% End: 
