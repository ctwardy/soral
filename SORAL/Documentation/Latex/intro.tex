%%%%%%%%%%%%%%%%%%%%%%%%%%%%%%%%%%%%%%%%%%%%%%%%%%%%%%%%%
% intro.tex
% Introduction to allocation tech report in master.tex
% This section: Charles Twardy
%%%%%%%%%%%%%%%%%%%%%%%%%%%%%%%%%%%%%%%%%%%%%%%%%%%%%%%%%

In this Tech Report we present the resource allocation problem for
Search \& Rescue, review some of the literature, and describe some of
the algorithms. The intent of this report is to be the working
reference document for the development of an open-source
implementation of a resource allocation library. It is also of use to
programmers developing programs which use the resource allocation
library. The appendices may be of interest to those working in land Search \&
Rescue, as we relate the mathematical formulation of the problem to
the familiar terms used in NASAR\footnote{National Association of
  Search And Rescue, the prominent international association for
  ground SAR.} courses such as FUNSAR\footnote{Fundamentals of Search
  And Rescue} and MLPI\footnote{Managing the Lost Person Incident}.



\section{Introduction}
\label{sec:intro}
Informally, the problem at hand is to find a stationary lost object or
person as quickly as possible. That is, we want to allocate our
resources so that our average, or expected, time to detection is a
minimum. So our ultimate goal is to minimize our mean time to
detection.

This goal is difficult to pursue directly, since we usually do not
know how many resources we will have in the future. Therefore, we
usually proceed in stages. At each stage we have a pool of resources
available, and we seek to use them most efficiently. That is, at every
stage we try to maximize our chances of finding the target. Often this
strategy also results in the lowest mean time to detection.

Formally, then, we have a constrained minimization problem. The good
news is that with a little mathematics, we can write down in one
equation the function we need to minimize. However, we already know
the answer: the function we need has a minimum when we apply an
infinite amount of effort to all areas. So the bad news is that we
have constraints. First, we don't have an infinite amount of effort,
so the total effort from any resource must add up to its total
available resource-hours. Second, we cannot apply negative effort to
any area! Although the second constraint is obvious in the real world,
we have to specify it explicitly because mathematical functions are
defined for negative numbers too.\footnote{Indeed, if the real answer
  leaves area $a$ unsearched, we could do quite a bit better if we
  could spend negative time in $a$, and then use some of that extra
  time in some other area!}


\subsection{Problem Setup}
\label{sec:problem-setup}

A crucial first step is to define a possibility area. By definition,
the target is assumed to be somewhere in the possibility
area.\footnote{Sometimes it is convenient to include a fake area
  called ``Rest Of World'' in the possibility area. It is not
  necessary to do this, and we will not. It is easy to add the ``ROW''
  area to our algorithm if desired. And ``ROW'' is a nice simple way
  to express confidence in the scenario itself. A better, but more
  complicated solution, is to have alternate scenarios, each with
  different possibility areas, and a probability distribution over the
  alternate scenarios. ``ROW'' is a more practical alternative.
  Usually it is enough just to realize that if you have thoroughly
  covered your possibility area without finding your target or clues,
  you should expand your possibility area.} Second, the possibility
area is divided into individual regions or areas according to some
practical considerations about where the target is likely to be. Given
what we know about our target, the land, and the situation, each
region will have some probability of containing the target. For any
lost person, that probability will vary according to distance, size,
and type of terrain or vegetation in the region. Although that initial
probability assignment is itself quite difficult, we will presume that
it has been done, and that it is correct.

For each resource type, we need to know the amount of effort available
(in resource-hours, for example). We also need to know how effective
that resource is at finding targets like our on each type of area in
our possibility area. For example, bicycles are fast and
effective on roads and tracks, but not very good in thick
vegetation. Airborne infrared scopes are very good over open areas,
and very poor over forests. What we would like for each resource type
is a table showing that resource's travel speeds and \emph{sweep
  widths} for each combination of terrain, vegetation, conditions, and
target. We will never have all of that, but likely we can get data for
standard conditions, and some correction factors for nonstandard
conditions. 

Our problem then is to allocate \emph{all available effort} among all
the areas in such a way as to maximize our probability of detecting
the target. We will make the idealization that we can divide our
effort as finely as we wish, and that no time is spent in transit. 

\subsection{Notation}

We have chosen to make the notation as minimal as possible. In
Appendix 1 we show how our notation is related to other notation
commonly used in ground SAR. We need only five basic concepts:
\begin{itemize}
\item $a: 1\ldots A$ indexes areas from 1 to $A$.
\item $r: 1\ldots R$ indexes resources from 1 to $R$.
\item $p_a$ is the probability of finding the lost person in area $a$
\item $w_{ar}$ is a measure of the effectiveness of resource $r$ in
  area $a$.
\item $\Pr(D)$ is the total probability of detecting the subject. We
  seek to maximize this quantity.
\end{itemize}
We can define a few extra concepts for convenience, such as
$\Pr(\no{D}) = 1 - \Pr(D)$, the total probability of \emph{not}
detecting the subject. If you have a method of \emph{minimizing} a
function, $\Pr(\no{D})$ is what you want to minimize.


%%% Local Variables: 
%%% mode: latex
%%% TeX-master: "master"
%%% End: 
