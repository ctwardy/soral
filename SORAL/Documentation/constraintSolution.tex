\documentclass[10pt]{article}
\input{crtdefs}
\afourSetup{}
\title{Optimal Resource Allocation for Search \& Rescue}
\author{dwa, crt, AGM}

\begin{document}
%\abstract{\textsc{draft} An optimal solution to the problem of
%  allocating multiple search resources to multiple areas, where each
%  resource may have a different effectiveness in each area. We assume
%  all resources use the random (exponential) detection function, as 
%  is appropriate for ground searchers.}
\maketitle

In searching for a lost person, we usually have far fewer resources
than needed to search the whole possibility area thoroughly. And it is
important that we find the person as quickly as possible. In order to
do this, we need not to waste any effort. We need to find an
\emph{optimal} allocation of various resources at hand to the area at
hand. A crucial first step is to divide the whole possibility area
into a number of smaller areas or regions of known size, and assign to
each region a probability of containing the subject. Ideally we have
divided the possibility area finely enough so that each region denotes
one particular type of terrain and vegetation, and all parts of that
region have the same probability density. The solution presented here
assumes this has been done.

Once the area has been divided into regions, we need to look at our
resources. For each type of resource we need the amount of effort
available (in resource-hours, for example). We also need to know how
effective each type of resource is in each region. To this end, most
likely what we will have is a set of travel speeds and sweep widths
for each resource in standard conditions, and some modifying factors
which can be applied given the terrain, vegetation, weather, etc.

Given this information a computer may quickly find an optimal
allocation of resources to areas. Because we have represented our
effort abstractly in terms of resource-hours, and because we are doing
our allocation as a block, our optimal solution is what has been
called ``T-optimal'' \cite{}: no other allocation of \emph{all
  available} effort will yield a greater total probability of success.


\section{Previous solutions}
\label{sec:Previous-solutions}

Two restricted problems have been solved. The \emph{Charnes-Cooper
  algorithm} (\cite{}) solves the case of
allocating the effort of one single kind of resource across many
areas, where the effectiveness of the resource varies by area. The
solution is to sort the areas by a quantity known as the probable
success rate $(PSR)$.

Alternatively, if we ignore the fact that resource effectiveness
changes with region, then we can solve the problem for any number of
resources using a slight simplification of the Charnes-Cooper
algorithm: we sort the areas by probability-density. 

Both such solutions are very efficient. They run in O$(A^2)$ time,
meaning that the time to compute the solution goes up as the square of
the number of areas. Since we probably never have more than 1000 areas and 
1000 resources, any computer can solve the problem quite quickly.

In the general problem, however, we have multiple resources, all of
which may vary in effectiveness across regions. One way to approach
the problem is to divide all of our resources into teams ahead of
time, and then to find that assignment of teams to areas which
maximizes our probability of success. In essence we just try all
combinations. This is the approach taken by CASIE III (\cite{}). 

However, even if we add the restriction that we may have precisely one
resource to one area, the problem has a complexity of O$(A^R)$: the
time goes up as the exponential of the number of resources. Even a
moderate search of say 10 areas and 10 resources requires
10,000,000,000 steps. Even if solved, this solution is not an optimum
allocation of effort! It is merely an optimum given how we divided
up the available resource pool. Faced with that kind of combinatorial
explosion, the general problem of optimum resource allocation for
search has resisted solution.

However, if we allow the idealization that the available effort can be
divided arbitrarily finely, then we can use some optimization
techniques from linear algebra and linear programming to solve the
problem. It appears that a solution can be found with a complexity of
O$R(A+1)$ which is basically the number of resources times the number
of areas, or 100 in the previous example.

The key is to find the proper global function to optimize, and
conceptually to perform the optimization all at once, rather than
combinatorially.

\section{Deriving the Function}
\label{sec:Deriving-Function}

In order to derive the function we need to optimize, we have to
transform the problem from the language of search-and-rescue to a more 
convenient notation. In this section we do this twice. First, using
the traditional phrasing and notation of the search community. Second,
using the same ideas but with the cleaner notation of conditional
probability, which is really what the search problem is about. First
we define our variables.


\subsection{Background}
\label{sec:Background}

The basic equation governing search probability theory\footnote{There
  is no new material in this section. We are merely laying out search
  theory, as can be found in \cite{}, in a convenient manner for
  subsequent derivations.} states that the probability of success
$(POS)$ of any search assignment is equal to the product of two other
probabilities: the probability that the subject is in the area $(POA)$
and the probability of detecting the subject if they are in that area
$(POD)$.
\begin{equation}
  \label{eq:POS}
  POS = POA \times POD
\end{equation}
For now we shall just accept this. It can be derived from a few simple
considerations about search and Bayes' theorem, but we leave that to
the section on conditional probabilities.

Once an area is searched (without
success), the subject is less likely to be there, in proportion to how 
well the area was search. In other words, after a search, the $POA$ of 
the region searched decreases. It can also be shown that the usual
formula is correct:
\begin{equation}
  \label{eq:POA_new}
  POA_{new} = POA_{old} - POS
\end{equation}

It is easy to understand $POA$. Each region has a known area (in
square meters, for example), and a probability of area $POA$, the
probability that the subject is in that area.\footnote{Sometimes
  called the probability of containment, $POC$ since it is the
  probability that the area in question contains the subject.} The
$POA$ is usually determined by a Mattson consensus (a fancy name for
``average'') from case histories, known facts, and probable scenarios.
Here we will just assume that it is given and correct. Often we
combine these two items to define the \emph{probability density}:
\begin{equation}
  \label{eq:2}
  Pden = \frac{POA}{area}
\end{equation}
Sometimes $Pden$ is written as greek letter $\rho$, since it looks
like a funny `p'.

It is a little more involved to get the $POD$ of a resource in an
area. Although search managers usually use various crude estimation
techniques (``If there were 100 milk jugs in the area, how many do you 
think you would have found?''), search theory provides a better way to 
measure $POD$ based on how much effort was applied.

First, every resource can move with a certain speed $v$ in a certain
region and at that speed (in those conditions) have a certain
\emph{effective sweep-width}, usually denoted capital $W$.  $W$ is a
measure of how far that resource can expect to see and detect objects.
There are tables of sweep widths for many resources in many
conditions. We shall presume we have them available.\footnote{Alas,
  there are few or none for ground searchers yet. Until as a community 
  we can conduct some experiments, one of the authors will attempt to
  make reasonable estimates available perhaps using the testing data
  from SARTECH exams.}

Together $W$ and $v$ make up the \emph{effective sweep-rate}, $ESR = W
v$. $ESR$ measures how quickly the resource can sweep areas. We let
$x$ be our measure of how much effort (resource-hours) has been put in 
the region.

To get $POD$ we need to know how well the region has been
\emph{covered}. \emph{Coverage} is a measure of how thoroughly a given
amount of effort was able to search an area of a certain size. Higher
coverage means higher $POD$. For a given amount of a particular
resource's effort, a bigger area means less coverage. Likewise, for a
given amount of effort in a given area, a resource with a bigger $ESR$
will have more coverage. And finally, the longer we leave the resource
in a region (or the more of it we put there), the more area it sweeps,
and the higher the coverage. So, if we let $x$ be a measure of how
much of the resource's effort we put in an area, we may define a
resource's \emph{coverage}:
\begin{equation}
  \label{eq:C}
  C = \frac{ESR x}{area}
\end{equation}

And $POD$ must be a function of coverage. It must be such that $POD$
starts at zero when we have zero coverage, and $POD$ can 
never exceed 1 no matter how much we cover the area. One function
which can do this is a negative exponential function. But we're not
just picking functions out of the air. It turns out that this is the
proper function for what is called \emph{random search}, and that
almost any real ground search is best modeled by the random-search
curve. \cite{} Therefore,
\begin{equation}
  \label{eq:3}
  POD = 1 - e^{-C}
\end{equation}
which can also be written
\begin{equation}
  \label{eq:9}
  POD = 1 - exp(-C)
\end{equation}
(Since later on we will have many things in the exponent, it will be
easier to read if we use the `$exp$' form.)

\subsection{Derivation 1: SAR notation}
\label{sec:SAR-notation}

We have $R$ resources available for searching. Together, they give us
the total amount of available effort, $X$. The effort due to any
resource $r$ is $x_r$. It follows that 
\begin{equation}
  \label{eq:10}
  X = \sum_{r=1}^N x_r
\end{equation}

We also have $A$ areas to search. Each area $a$ has a certain
probability of area $POA_a$. Usually all probabilities initially sum
to 1, but since they do not really have to be normalized, the only
constraint is that: $(0 < \sum_{a=1}^A POA_a \leq 1)$.

Our problem is to figure out the best way to allocate all available
effort among the many areas. The effort  $x_r$ from any resource
may be split among any number of areas, but $r$ will always be used
completely, so $\sum_{a=1}^A x_{r,a} = x_r$. 

Conversely, several resources may combine their efforts in any one
area. But while there may be many areas left unsearched, we will never
fail to use a resource fully.

Coverage is defined in terms of resources and areas. If $ESR_{r,a}$ is the 
ESR for resource $r$ in area $a$, then the coverage in area $a$ by
resource $r$ is:

\begin{equation}
  \label{eq:C_ra}
  C_{r,a} = \frac{ESR_{r,a}}{A_a} x_{r,a}
\end{equation}

However, because the area sizes and the $ESR_{r,a}$ remain fixed,
rather than continuing to write $\frac{ESR_{r,a}}{A_a}$, we will
just define a new variable $w_{r,a} = \frac{w_{r,a}}{A_a}$ and write:
\begin{equation}
  \label{eq:C_ra}
  C_{r,a} = w_{r,a} x_{r,a}
\end{equation}

Once we know the coverage, we can find the $POD$. 
\begin{equation}
  \label{eq:POD_ra}
  POD_{r,a} = 1 - exp(-C_{r,a}) = 1 - exp(-w_{r,a} x_{r,a})
\end{equation}

Now in the background section, we wrote the governing equation of
search theory $POS = POA \times POD$, and the associated equation for
updating the $POA$ of a region after an unsuccessful search:
$POA_{new} = POA_{old} - POS$. 

Since the second equation is just subtraction, we can apply the
formula repeatedly, and define the $POA$ for any sortie in terms of
the original $POA$ and the sum of all the $POS$ from all previous
searches. If we number our searches with an index $i$:
\begin{equation}
  \label{eq:POA2'}
  POA_{i+1} = POA_0 - \sum_1^i POS_i
\end{equation}


This generalizes quite easily to our notation. If the initial $POA$
for area $a$ is just $POA_a$, then to compute the current $POA$ facing
our next resource $r$ in area $a$, we sum over the
previously-allocated resources (up to resource $r-1$)\footnote{
  The order in which we consider resources for assignment is
  immaterial: all we want in the end is the right global allocation.
  So we might as well pick the resources straight off the list in
  order $1 \ldots r$. Which is how we have done it here.}
:
\begin{equation}
  \label{eq:POA_ra}
  POA_{r,a} = POA_a - \sum_{i=1}^{r-1}POS_{i,a}
\end{equation}

Using the basic identity from equation \ref{eq:POS} ($POS =
POA \times POD$), we can
rewrite \ref{eq:POA_ra} in terms of products (see the appendix for derivation):
\begin{eqnarray}
  \label{eq:15}
  \nonumber POA_{r,a} &=& POA_a \left[(1-POD_{1,a})\cdots(1-POD_{r-1,a})\right]\\
                   & & \\
  \nonumber &=& POA_a \left[exp(w_{1,a}x_{1,a}) \cdots exp(w_{r-1,a}x_{r-1,a})\right]\\
  \nonumber &=& POA_a \left[exp\left(\sum_{i=1}^{r-1}w_{i,a}x_{i,a}\right)\right]
\end{eqnarray}
At this point we are almost done.

Putting the $POS_{r,a}$ and the $POA_{r,a}$ together:
\begin{equation}
  \label{eq:POS2_ra}
  POS_{r,a} = POA_{r,a} \cdot POD_{r,a}
\end{equation}
which again comes from equation \ref{eq:POS}. 

And, in the end, we want to \emph{maximize} the \emph{total}
$POS$. The total $POS$ is just the sum over both $r$ and $a$. Writing
this out and substituting for $POA_{r,a}$ we find:
\begin{eqnarray}
  \label{eq:POS_tot}
  POS &=& \sum_{a=1}^A \sum_{r=1}^R POS_{r,a}\\
  \nonumber &=& \sum_{a=1}^A POS_{1,a} + POS_{2,a} + \cdots + POS_{R,a}\\
  \nonumber &=& \sum_{a=1}^A \left[
    POA_a POD_{1,a} +
    POA_a exp(-w_{1,a}x{1,a}) POD_{2,a} + \cdots + 
    POA_a exp\left(-\sum_{i=1}^{R-1}w_{i,a}x_{i,a}\right) POD_{R,a} 
    \right]\\
   &=& \sum_{a=1}^A POA_a \left[
     POD_{1,a} +
     exp(-w_{1,a}x{1,a}) POD_{2,a}+ \cdots + 
     exp\left(-\sum_{i=1}^{R-1}w_{i,a}x_{i,a}\right) POD_{R,a}
     \right]
\end{eqnarray}

Finally, if we expand the $POD_{r,a}$ terms we find that many of the
exponents cancel, leaving us with a much simpler form.

\fbox{
  \parbox{6in}{
  \begin{eqnarray}
    \label{eq:pos-final}
    \nonumber POS &=& \sum_{a=1}^A POA_a \left[
      (1-exp(-w_{1,a}x_{1,a})) +
      exp(-w_{1,a}x{1,a}) (1-exp(-w_{2,a}x_{2,a}))+ \cdots +\right. \\
    & & \left. exp\left(-\sum_{i=1}^{R-1}w_{i,a}x_{i,a}\right)
      \left(1-exp(-w_{R,a}x_{R,a})\right)
    \right]\\
    &=& \sum_{a=1}^APOA_a \left[
      1 - exp\left(-\sum_{i=1}^Rw_{i,a}x_{i,a}\right)
    \right]    
  \end{eqnarray}
  }
}

%%%%%%%%%%%%%%%%%%%%%%%%%%%%%%%%%%%%%%%%%%%%%%%%%%%%
\subsection{Conditional Probability Notation}
\label{sec:Cond-Prob-Notat}

The initial probability (before we get new evidence) for anything is
called its \emph{prior probability}. For example, we know from the
start that we will use all of our resources. Therefore the prior
probability for any resource is 1:
\begin{equation}
  \label{eq:p(r)}
  \Pr(R) = 1
\end{equation}

Each area has an initial (prior) probability of containing the
subject, which is usually called the $POA$ in search terms. We will
call is $\Pr(A)$.
\begin{equation}
  \label{eq:p(a)}
  \Pr(A) = POA
\end{equation}

The probability of something given that we know something else is
called the \emph{conditional probability}. For example, we might wish
to know the conditional probability of an area $A$, given that we have
chosen a particular resource $R$. This one is easy: the probability of
an area does not depend upon which resource we use
to search it, so it equals its initial probability:
\begin{equation}
  \label{eq:p(a|r)}
  \Pr(A|R) = P(A)
\end{equation}

For a resource searching in the area containing the subject, the
probability of detecting the subject is given by search
theory\footnote{This is the correct formula for the ``random'' or
  ``exponential'' detection profile, which is shown in \cite{} to be
  the most appropriate model for ground search.}
as $1 - e^{-C}$. Keeping with our notation in the previous section, we
use $x_{r,a}$ to denote the amount of effort from resource $r$ used in
area $a$, and $w_{r,a}$ to denote the effectiveness of the resource in
that area. So:
\begin{equation}
  \label{eq:4}
  \Pr(D|R,A) = 1 - e^{-w_{r,a} x_{r,a}}
\end{equation}

So what is the \emph{posterior} probability of an area? It is the
probability that the subject is in the area given that we have
searched but \emph{not detected} ($\no{D})$ the subject. If we have
$N$ resources, then by Bayes'
theorem (without renormalization):
\begin{eqnarray}
  \label{eq:a|-d}
  \Pr(A|\no{D}) &\propto& \Pr(A) \cdot \Pr(\no{D}|A)\\
  \nonumber     &=& \Pr(A) \cdot \left[\Pr(\no{D}|R_1,A) \ldots 
                   P(\no{D}|R_N,A)\right] \\
  \nonumber     &=& \Pr(A) \left[\Pr(\no{D}|R_1,A) \ldots
                   P(\no{D}|R_N,A)\right] \\
  \nonumber     &=& \Pr(A) \left[e^{-w_{1,a} x_{1,a}} \ldots
                   e^{-w_{N,a} x_{N,a}} \right]\\
  \nonumber     &=& \Pr(A) exp\left(-\sum_{k=1}^N w_{k,a} x_{k,a}\right)
\end{eqnarray} 

That gives us the updated probability for one area. And to get the
total probability that we have \emph{not} detected the subject up
until now, we just sum over the areas. By definition, this is just the
sum over areas of the \emph{joint probability} of being in each area
and being not-detected:
\begin{equation}
  \label{eq:6}
  \Pr(\no{D}) = \sum_A \Pr(\no{D},A)
\end{equation}
But the joint probability of $\no{D}$ and $A$ is by definition: 
\begin{equation}
  \label{eq:8}
  \Pr(\no{D},A) =  \Pr(A) \Pr(\no{D}|A)
\end{equation}
which is just the right-hand-side of Equation \ref{eq:a|-d}. We
infer that:
\begin{equation}
  \label{eq:-d|a}
  \Pr(\no{D}|A) = exp\left(-\sum_{k=1}^N w_{k,a} x_{k,a}\right)
\end{equation}

Of course, in the end we want to maximize our chances of \emph{actually
  detecting} the subject, for the \emph{whole set of
  assignments}. Usually called the ``probability of success,'' this is
really just the overall $\Pr(D)$.

\fbox{
  \parbox{6in}{
    \begin{eqnarray}
      \label{eq:p(d)}
      \nonumber \Pr(D) &=& \sum_A \Pr(A) \Pr(D|A)\\
      \nonumber  &=& \sum_A \Pr(A) \left[ 1 - \Pr(\no{D}|A) \right]\\
      & & \\
      \nonumber  &=& \sum_A \Pr(A) \left[ 1 - exp\left(-\sum_{i=1}^N w_{i,a}
          x_{i,a}\right) \right]
    \end{eqnarray}
    }
  }

Note that this agrees with Equation \ref{eq:pos-final}, the overall
$POS$ derived using conventional notation, except that we have
replaced the letter $R$ with the letter $N$. In the rest of the paper
we will go back to using $R$.

\fbox{actually it wd be better to replace `R' with `r' and `N' with `R'}


\section{Optimizing the Allocation}
\label{sec:Optim-Alloc}

\fbox{ Need to update this section.}

But since $\Pr(D) = 1 - \Pr(\no{D})$, it follows that we
should \emph{minimize} our overall chance of missing the subject:
$\Pr(\no{D})$.

    \begin{eqnarray}
      \label{eq:7}
      \Pr(\no{D}) &=& \sum_A P(\no{D},A)\\
      \nonumber &=& \sum_A \Pr(A) e^{-\sum_{k=1}^N S_{k,a} x_{k,a}}
    \end{eqnarray}

Another way to get the same answer is to maximize the cumulative
$POS_A = P(A) - P(A|\no{D})$. It comes out to the same thing, and is
more familiar to search managers.

Fine, now we know what function we need to optimize. How do we do it?

I have some very good news. To find the solution to your problem only
requires solving $R + RA = R(A+1)$ linear equations, where $R$ is the
number of resources and $A$ is the number of areas.

I will give you details on Friday, if you are interested (I will not be in
tomorrow). However, an outline goes as follows.

Your problem can be stated as find $x_{i,j} >= 0$ which minimize

\begin{equation}
f = \sum_{i=1}^A p_i*g_i
\end{equation}
where

\begin{equation}
  g_i = exp(-\sum_{k=1}^R w_{i,k}*x_{i,k}), \;\; i=1,\ldots,A
\end{equation}

and we also know that:

\begin{equation}
  \sum_{i=1}^A x_{i,j} = T_j, \;\; j=1,\ldots,R   
\end{equation}

Now, let

\begin{equation}
   L = \sum_{i=1}^A p_i*g_i + \sum_{j=1}^R L_j*(\sum_{i=1}^A x_{i,j}-T_j)
\end{equation}

Then the problem can be reformulated as find the critical points of $L$ (ie
those values of $x_{i,j}$ and $L_j$ where the partial derivatives of $L$ are
zero), where $x_{i,j}$ are positive.

However, the partial derivative of $L$ with respect to $x_{i,j}$ is zero when
\begin{equation}
         p_i*g_i*w_{i,j} = L_j
\end{equation}

Taking logs, and letting $c_j = log(L_j), j = 1,\ldots,R$, we get
\begin{equation}
         c_j + \sum_{k=1}^R w_{i,k}*x_{i,k} = log(p_i) + log(w_{i,j})
\end{equation}

for $i = 1,\ldots,A$ and $j = 1,\ldots,R$. Also, we have the equations
\begin{equation}
         \sum_{i=1}^A x_{i,j} = T_j, j=1,\ldots,R
\end{equation}

So, all we need to do is solve these $R(A+1)$ linear equations for $c_j$ and
$x_{i,j}$.

Which of course we are now working on.

\end{document}

%%% Local Variables: 
%%% mode: latex
%%% TeX-master: t
%%% End: 
