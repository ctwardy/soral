%%%%%%%%%%%%%%%%%%%%%%%%%%%%%%%%%%%%%%%%%%%%%%%%%%%%%%%%%
% related.tex
% Introduction to allocation tech report in master.tex
% This section: Charles Twardy
%%%%%%%%%%%%%%%%%%%%%%%%%%%%%%%%%%%%%%%%%%%%%%%%%%%%%%%%%

\section{Related Work}
\label{sec:related}

Although the problem of optimal resource allocation is only beginning
to be known in the land search community, it has been studied in the
context of maritime search since World War II when the U.S. Navy
needed to find enemy submarines. There has been quite a bit of
mathematical and computational theory in the Operations Research
literature. 

The general problem of optimal resource allocation can be broken down
according to whether there is a single resource or multiple resources,
whether the target is moving, whether the target is actively evading,
whether there are false detections, whether there is one or several
kinds of targets being sought, and whether the sweep width is
uncertain. Stone \cite{} provides a nice overview, as does ....

In time this paper might expand to explain several algorithms, but it
is mostly concerned with the case of a \emph{single,
  stationary} target with \emph{multiple} resources to be allocated,
no false detections, and no uncertainty in the sweep width. Although
we have been working on an algorithm, that project is on hold. The
algorithm we will detail in later sections was developed by Alan Washburn
\cite{washburn:_jota}. 

Here is a brief survey of other methods.

\subsection{Literature}
The easiest introduction to the mathematical problem is Washburn's
short textbook, \emph{Search \& Detection} \cite{}, in particular the
beginning of chapter 2, and chapters 4 \& 5. The section on Lagrangian
multipliers uses Everett's theorem, which greatly simplifies
matters. Further complications such as target motion are
considered separately in later chapters. However, the text does not
consider the multiple-resources case, even though the original version
of Everett's theorem \cite{} handles multiple resources.

Most ground searchers are not likely to make it past the word
``Lagrangian''. The easiest (and really, only) introduction for
ordinary ground searchers is the four-part series \emph{*****} by
J.R. Frost \cite{}. Frost spends the first three sections laying out
the foundations of sweep width so that finally in the fourth section
he can discuss the Charnes-Cooper algorithm. Although Frost nicely
avoids complicated mathematics, he is laboring under a handicap: his
audience (including one of the present authors) has been using poor
substitutes for about three decades, in ignorance of the existence of
actual theory. Since Frost has already put in an effort for
un-teaching, we will start with the basic elements of the problem and
hopefully therefore have a more straightforward presentation.

The most complete survey is Stone's \emph{Theory of Optimal Search}
\cite{stone89:_optimal}. Stone is responsible for many of the major
advances in search theory since it's original formulation by Koopman
in the 1940s, and his book is an excellent reference and a good State
of the Art report. However, after the initial sections it is quite
densely mathematical. 

Those particularly interested in the application of search theory
should look at \emph{*********}, the proceedings of a NATO conference
on search theory in 19**. In addition to presenting some developments
following the first edition of Stone's book, the papers in that volume
explore various applications and computer systems.


We turn now to a closer look at some particular solutions.

\subsection{Two restricted solutions: stationary target}
\label{sec:previous}

Stone \cite[81]{stone89:_optimal} surveys previous solutions to this
problem. In summary, the problem was presented in Koopman's original
work for the U.S. Navy in World War II. Koopman solved the problem for
the circular normal target distribution in 1946, and for other
distributions in 1957. Charnes and Cooper presented an algorithm for
finding that solution in 1958. Stone unified the previous work in
the first edition of his book, in 1975, and showed how to derive a
more general Charnes-Cooper algorithm from the analytic solution.

The Charnes-Cooper algorithm sorts the areas by a quantity known as
the probable success rate ($PSR = w_{r,a} / POA$), and applies effort
to level the $PSR$ field. This algorithm is one way to find the
solutions provided by analytic optimization methods like Lagrange
multipliers. Another such algorithm is offered in Washburn
\cite{} based on Everett's theorem. 

Another restricted problem is the case where we ignore the fact that
resource effectiveness changes with area. That is, for each
resource $r$, its $w_{r,a}$ is the same for all $a$. So all we have
are $r$ different factors $w_r$. We can solve this problem for any
number of resources using a slight simplification of Charnes-Cooper:
we sort the areas by probability density.

Both solutions are very efficient: they run in O$(A^2)$ time. However,
neither solution can be applied to the case of multiple resources
which vary in their effectiveness in different areas. Although we
could just break up our map into areas and try all combinations of
areas and resources. The software program CASIE III, widely used in
land search, does just this. Even with the restriction that we may
have precisely one resource to one area, the problem has a complexity
of O$(A^R)$: even a search with 10 resources and 10 areas requires
10,000,000,000 steps, and the resulting solution is not strictly
optimal.


\subsection{Multiple resources: stationary target}
\label{sec:mult}

Washburn \cite{washburn95:_jota} has presented a graph-theoretic
algorithm which solves the multiple resource allocation
problem in O$(R^{3/2}A^{3/2})$, according to an empirical test. This
is quite good. Even a 100$\times$100 problem can be done in 1 million
steps, which is to say almost immediately.

We detail this algorithm in later sections. Washburn's paper requires
close attention and several readings. We hope to save readers that
effort. 

\subsection{Moving targets}
\label{sec:moving}

Of course, in a real search, the target is likely to be moving, at
least for awhile. We do not consider targets which are
evading.\footnote{We should consider the case of evasive targets when
  we are on a manhunt, or searching for psychotic or suicidal
  subjects, for some Alzheimer's patients, and for children old enough
  to know not to talk to strangers, but young enough not to be able to
  tell that this might be an OK time to do so.}

Scott Brown \cite{brown80:_optimal} proved in 1978 that,
\begin{quotation}
  a search plan maximizes overall probability of detection if and only
  if for each time interval $i$ the search conducted at time $i$
  maximized the probability of detecting a stationary target\ldots.
\end{quotation}
However, Brown's proof was for the case of a single resource.

Washburn \cite{washburn83:_fab} developed  algorithm to handle the
case with multiple resources and a moving target. Washburn's FAB
algorithm repeatedly calls some stationary target algorithm, so our
solution could be used if it proved faster than Washburn's.


\subsection{Clues and false detections}
\label{sec:false}


%%% Local Variables: 
%%% mode: latex
%%% TeX-master: "master"
%%% End: 
